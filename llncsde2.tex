% This is LLNCSDE2.TEX, a variation of LLNCS.DEM
% (the demonstration file of
% the LaTeX macro package from Springer-Verlag
% for Lecture Notes in Computer Science,
% version 2.3 for LaTeX2e),
% which can be used by volume editors for the preparation
% of the front matter pages and the author index
%
% Last changes: 16.10.2006, Frank Holzwarth (texhelp@springer.de)
%
%%%%%%%%%%%%%%%%%%%%%%%%%%%%%%%%%%%%%%%%%%%%%%%%%%%%%%%%%%%%%%%%%%%%%
% In order to generate an Author Index do the following:
% After TeXing this document start the program MakeIndex by typing
% MAKEINDX -S SPRMINDX.STY <filename>
% (generates an IND file for the Author Index)
% into the DOS command line.
% (At other systems you may have to use the command MAKEINDEX.)
% Now TeX this file once again, then you will get an Author Index.
% TeX this file once more, then the TOC will be complete.
%%%%%%%%%%%%%%%%%%%%%%%%%%%%%%%%%%%%%%%%%%%%%%%%%%%%%%%%%%%%%%%%%%%%%

\documentclass{llncs}
%
\usepackage{makeidx}  % allows for indexgeneration
\makeindex
%
\begin{document}
%
\frontmatter          % for the preliminaries
\setcounter{page}{5}
%
\pagestyle{headings}  % switches on printing of running heads
\addtocmark{Hamiltonian Mechanics} % additional mark in the TOC
%
\chapter*{Preface}
%
This textbook is intended for use by students of physics, physical
chemistry, and theoretical chemistry. The reader is presumed to have a
basic knowledge of atomic and quantum physics at the level provided, for
example, by the first few chapters in our book {\it The Physics of Atoms
and Quanta}. The student of physics will find here material which should
be included in the basic education of every physicist. This book should
furthermore allow students to acquire an appreciation of the breadth and
variety within the field of molecular physics and its future as a
fascinating area of research.

For the student of chemistry, the concepts introduced in this book will
provide a theoretical framework for that entire field of study. With the
help of these concepts, it is at least in principle possible to reduce
the enormous body of empirical chemical knowledge to a few basic
principles: those of quantum mechanics. In addition, modern physical
methods whose fundamentals are introduced here are becoming increasingly
important in chemistry and now represent indispensable tools for the
chemist. As examples, we might mention the structural analysis of
complex organic compounds, spectroscopic investigation of very rapid
reaction processes or, as a practical application, the remote detection
of pollutants in the air.

\vspace{1cm}
\begin{flushright}\noindent
April 1995\hfill Walter Olthoff\\
Program Chair\\
ECOOP'95
\end{flushright}
%
\chapter*{Organization}
ECOOP'95 is organized by the department of Computer Science, Univeristy
of \AA rhus and AITO (association Internationa pour les Technologie
Object) in cooperation with ACM/SIGPLAN.
%
\section*{Executive Committee}
\begin{tabular}{@{}p{5cm}@{}p{7.2cm}@{}}
Conference Chair:&Ole Lehrmann Madsen (\AA rhus University, DK)\\
Program Chair:   &Walter Olthoff (DFKI GmbH, Germany)\\
Organizing Chair:&J\o rgen Lindskov Knudsen (\AA rhus University, DK)\\
Tutorials:&Birger M\o ller-Pedersen\hfil\break
(Norwegian Computing Center, Norway)\\
Workshops:&Eric Jul (University of Kopenhagen, Denmark)\\
Panels:&Boris Magnusson (Lund University, Sweden)\\
Exhibition:&Elmer Sandvad (\AA rhus University, DK)\\
Demonstrations:&Kurt N\o rdmark (\AA rhus University, DK)
\end{tabular}
%
\section*{Program Committee}
\begin{tabular}{@{}p{5cm}@{}p{7.2cm}@{}}
Conference Chair:&Ole Lehrmann Madsen (\AA rhus University, DK)\\
Program Chair:   &Walter Olthoff (DFKI GmbH, Germany)\\
Organizing Chair:&J\o rgen Lindskov Knudsen (\AA rhus University, DK)\\
Tutorials:&Birger M\o ller-Pedersen\hfil\break
(Norwegian Computing Center, Norway)\\
Workshops:&Eric Jul (University of Kopenhagen, Denmark)\\
Panels:&Boris Magnusson (Lund University, Sweden)\\
Exhibition:&Elmer Sandvad (\AA rhus University, DK)\\
Demonstrations:&Kurt N\o rdmark (\AA rhus University, DK)
\end{tabular}
%
\begin{multicols}{3}[\section*{Referees}]
V.~Andreev\\
B\"arwolff\\
E.~Barrelet\\
H.P.~Beck\\
G.~Bernardi\\
E.~Binder\\
P.C.~Bosetti\\
Braunschweig\\
F.W.~B\"usser\\
T.~Carli\\
A.B.~Clegg\\
G.~Cozzika\\
S.~Dagoret\\
Del~Buono\\
P.~Dingus\\
H.~Duhm\\
J.~Ebert\\
S.~Eichenberger\\
R.J.~Ellison\\
Feltesse\\
W.~Flauger\\
A.~Fomenko\\
G.~Franke\\
J.~Garvey\\
M.~Gennis\\
L.~Goerlich\\
P.~Goritchev\\
H.~Greif\\
E.M.~Hanlon\\
R.~Haydar\\
R.C.W.~Henderso\\
P.~Hill\\
H.~Hufnagel\\
A.~Jacholkowska\\
Johannsen\\
S.~Kasarian\\
I.R.~Kenyon\\
C.~Kleinwort\\
T.~K\"ohler\\
S.D.~Kolya\\
P.~Kostka\\
U.~Kr\"uger\\
J.~Kurzh\"ofer\\
M.P.J.~Landon\\
A.~Lebedev\\
Ch.~Ley\\
F.~Linsel\\
H.~Lohmand\\
Martin\\
S.~Masson\\
K.~Meier\\
C.A.~Meyer\\
S.~Mikocki\\
J.V.~Morris\\
B.~Naroska\\
Nguyen\\
U.~Obrock\\
G.D.~Patel\\
Ch.~Pichler\\
S.~Prell\\
F.~Raupach\\
V.~Riech\\
P.~Robmann\\
N.~Sahlmann\\
P.~Schleper\\
Sch\"oning\\
B.~Schwab\\
A.~Semenov\\
G.~Siegmon\\
J.R.~Smith\\
M.~Steenbock\\
U.~Straumann\\
C.~Thiebaux\\
P.~Van~Esch\\
from Yerevan Ph\\
L.R.~West\\
G.-G.~Winter\\
T.P.~Yiou\\
M.~Zimmer\end{multicols}
%
\section*{Sponsoring Institutions}
%
Bernauer-Budiman Inc., Reading, Mass.\\
The Hofmann-International Company, San Louis Obispo, Cal.\\
Kramer Industries, Heidelberg, Germany
%
\tableofcontents
%
\mainmatter              % start of the contributions

% first contribution

\title{Hamiltonian Mechanics unter besonderer Ber\"ucksichtigung der
h\"ohreren Lehranstalten}
%
%\titlerunning{Hamiltonian Mechanics}  % abbreviated title (for running head)
%                                     also used for the TOC unless
%                                     \toctitle is used
%
\author{Ivar Ekeland\inst{1} \and Roger Temam\inst{2}
Jeffrey Dean \and David Grove \and Craig Chambers \and Kim~B.~Bruce \and
Elisa Bertino}
%
%\authorrunning{Ivar Ekeland et al.}   % abbreviated author list (for running head)
%
%%%% list of authors for the TOC (use if author list has to be modified)
\tocauthor{Ivar Ekeland, Roger Temam, Jeffrey Dean, David Grove,
Craig Chambers, Kim B. Bruce, Elisa Bertino}

\index{Ekeland, Ivar}
\index{Temam, Roger}
\index{Dean, Jeffrey}
\index{Grove, David}
\index{Chambers, Craig}
\index{Bruce, Kim B.}
\index{Bertino, Elisa}
% use the command \index{<name>} for index entries

%
\institute{Princeton University, Princeton NJ 08544, USA,\\
\email{I.Ekeland@princeton.edu},\\ WWW home page:
\texttt{http://users/\homedir iekeland/web/welcome.html}
\and
Universit\'{e} de Paris-Sud,
Laboratoire d'Analyse Num\'{e}rique, B\^{a}timent 425,\\
F-91405 Orsay Cedex, France}

\maketitle              % typeset the title of the contribution

\clearpage

% second contribution

\setcounter{page}{7}

\title{Hamiltonian Mechanics2}

\author{Ivar Ekeland\inst{1} \and Roger Temam\inst{2}}

\index{Ekeland, Ivar}
\index{Temam, Roger}
% use the command \index{<name>} for index entries

\institute{Princeton University, Princeton NJ 08544, USA
\and
Universit\'{e} de Paris-Sud,
Laboratoire d'Analyse Num\'{e}rique, B\^{a}timent 425,\\
F-91405 Orsay Cedex, France}

\maketitle

\clearpage

\setcounter{page}{13}

\addtocmark[2]{Author Index} % additional numbered TOC entry
\renewcommand{\indexname}{Author Index}
\printindex

\end{document}
